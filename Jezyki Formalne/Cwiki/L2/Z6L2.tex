\documentclass{article}
\usepackage[T1]{fontenc}
\usepackage[utf8]{inputenc}
\usepackage{lmodern}
\usepackage[polish,shorthands=off]{babel}
\usepackage{amsmath}
\usepackage{amssymb}
\title{Zadanie 6 z listy 2}
\author{Jakub Kowal}
\date{}
\begin{document}
    \maketitle
    \section*{Polecenie}

    \begin{quote}
        Udowodnij, że jeśli dla pewnego języka L istnieje niedeterministyczny automat skończony
rozpoznający go, to istnieje również niedeterministyczny automat skończony rozpoznający
język $L^{R} = \{w : w^{R} \in L\}$.
    \end{quote}
    \section*{Rozwiązanie}
    Zakładamy, że $L\subseteq \Sigma*$ jest rozpoznawalny przej jakiś NFA. Zdefiniujmy ten NFA jako:
    \begin{center}
        $N = (Q,\Sigma,\delta,q_{0},F)$
    \end{center}
    Wtedy NFA rozpoznające $L^{R}$ wygląda następująco:
    \begin{center}
        $N^{R} = (Q\cup \{q_{s}\},\Sigma,\delta^{R},q_{s},q_{0})$
    \end{center}
    gdzie $\delta^{R}$ jest odwróceniem $\delta$, tj dla każdego przejścia $q_{i} \rightarrow q_{j}$ w $\delta$ $\delta^{R}$ ma przejścia odwrotne: $q_{j} \rightarrow q_{i}$. Warto też dodać, że w $N^{R}$ muszą istnieć $\epsilon$ przejścia z $q_{s}$ do stanów $F$ z $N$.
    \begin{center}
        $\forall_{w \in L^{R}}w^{R}\in L$
    \end{center}
    Wtedy istnieją przejścia z $q_{0}$ do $F$ $\hat{q}(q_{0},w^{R})$, czyli $q_{0}\xrightarrow{w_{0}}q_{i}\xrightarrow{w_{2}}\ldots\xrightarrow{w_{n}}q_{F}\in F$. Odwracając te przejścia ($\delta^{R}$) uzyskujemy $q_{F}\xrightarrow{w_{n}}q_{j}\xrightarrow{w_{n-1}}\ldots\xrightarrow{w_{0}}q_{0}$, biorąc pod uwagę $\epsilon$ przejście z $q_{s}$ do $q_{F}$ widzimy, że jest to ścieżka w $N^{R}$. Oznacza to, że $N^{R}$ jest niedeterministycznym automatem rozpoznającym $L^{R}$.
    \begin{flushright}
        $\Box$
    \end{flushright}
\end{document}