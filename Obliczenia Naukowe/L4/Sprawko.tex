\documentclass{article}
\usepackage[T1]{fontenc}
\usepackage[utf8]{inputenc}
\usepackage{lmodern}
\usepackage[polish,shorthands=off]{babel}
\usepackage{graphicx}
\usepackage{siunitx}
\usepackage{multirow}
\usepackage{longtable}
\usepackage{hyperref}
\usepackage{float}
\usepackage{enumitem}
\title{Sprawozdanie obliczenia naukowe\\Lista 4}
\author{Jakub Kowal}
\date{}
\begin{document}
\maketitle

\section*{1--3 Zadania 1--3}
\label{sec:tasks1-3}
\subsection*{Opis zadania}
W zadaniach od 1 do 3 trzeba zaimplementować funkcje:\\
\begin{enumerate}
    \item Obliczającą ilorazy różnicowe.
    \item Obliczającą wartość wielomianu interpolacyjnego stopnia n w postaci Newtona w punkcie $x=t$ za pomocą algorytmu Hornera.
    \item Wyznaczającą współczynniki postaci naturalnej wielomianu interpolacyjnego.
\end{enumerate}
\subsection*{Implementacje}
\begin{enumerate}
    \item W tym zadaniu wykorzystałem rekurencyjne obliczanie ilorazów różnicowych w taki sam sposób jak występowało to na \href{https://cs.pwr.edu.pl/zielinski/lectures/scna/scnaw6.pdf#page=16}{slajdach z wykładu}. 
    \item Zadanie 2 wykorzystuje wzór na algorytm Hornera podany w \href{https://cs.pwr.edu.pl/zielinski/lectures/scna/scnal4cw.pdf#page=2}{zadaniu 8 na liście 4}.
\end{enumerate}
\setcounter{section}{3}

\section{Zadanie 4}
\label{sec:task4}
Some content for task 4.

\section{Zadanie 5}
\label{sec:task5}
Some content for task 5.

\section{Zadanie 6}
\label{sec:task6}
Some content for task 5.


\end{document}