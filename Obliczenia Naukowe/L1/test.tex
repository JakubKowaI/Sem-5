\documentclass{article}
\usepackage[T1]{fontenc}
\usepackage[utf8]{inputenc}
\usepackage{lmodern}
\begin{document}
\section{Zadanie 1}
Opis problemu:
W zadaniu należało wyznaczyć następujące wartości:
\begin{itemize}
        % \item $pc_{32}$ - suma iloczynów odpowiadających sobie
    \item $Macheps$ --- najmniejsza liczba $\mathrm{macheps}>0$ taka, że $\mathrm{fl}(1.0+\mathrm{macheps})>1.0$ i $\mathrm{fl}(1.0+\mathrm{macheps})=1+\mathrm{macheps}$
    \item $eta$ --- najmniejsza dodatnia liczba zmiennoprzecinkowa (najmniejsza liczba nieznormalizowana --- $\mathrm{MIN}_{\mathrm{sub}}$)
    \item liczba $MAX$ --- największa liczba nieznormalizowana --- $\mathrm{MAX}_{\mathrm{sub}}$
\end{itemize}
Oraz odpowiedzieć na pytania:
\begin{itemize}
    \item Jaki związek ma liczba $macheps$ z $precyzją$ $arytmetyki$ (oznaczaną na wykładzie przez $\epsilon$)?
    \item Jaki związek ma liczba $eta$ z liczbą $\mathrm{MIN}_{\mathrm{sub}}$?
    \item Co zwracają funkcje floatmin(Float32) i floatmin(Float64) i jaki jest związek zwracanych wartości z liczbą $\mathrm{MIN}_{\mathrm{nor}}$?
\end{itemize}
\subsection{test}
\section{Zadanie 2}
This is the second section.
\end{document}