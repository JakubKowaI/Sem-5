\documentclass{article}
\usepackage[T1]{fontenc}
\usepackage[utf8]{inputenc}
\usepackage{lmodern}
\usepackage[polish,shorthands=off]{babel}
\usepackage{graphicx}
\usepackage{siunitx}
\usepackage{multirow}
\usepackage{longtable}
\usepackage{hyperref}
\usepackage{float}
\title{Sprawozdanie obliczenia naukowe\\Lista 3}
\author{Jakub Kowal}
\date{}
\begin{document}
\maketitle

\section*{Zadania 1 - 3}
\subsection*{Opis problemu}
Zaimplementować metody znajdowania pierwiastków funkcji i sprawdzić je dla $f(x)=0$.\\
Implementowane metody:\\
\begin{itemize}
    \item Metoda bisekcji
    \item Metoda stycznych (Newtona)
    \item Metoda siecznych
\end{itemize}
\subsection*{Wyniki}
\begin{table}[H]
    \centering
    \begin{tabular}{|c|c|}
        \hline
        Metoda & Wynik \\&(r,f(r),it,err)\\
        \hline
        Bisekcji & 0, 0, 0, 1\\
        \hline
        Stycznych & 1.0, 0.0, 0, 0\\
        \hline
        Siecznych & NaN, 0.0, 1, 0\\
        \hline
    \end{tabular}
    \caption{Wyniki dla zadań 1-3}
    \label{tab:Z1-3}
\end{table}
\subsection*{Wnioski}
W metodzie bisekcji otrzymujemy sygnalizację błędu oznaczającą brak miejsca zerowego w podanych zakresie, ale wynika to ze sprawdzenia warunku $sgn(u)==sgn(v)$. Warunek ten dla $f(x)=0$ będzie prawdziwy. W przypadku metody stycznych mamy warunek sprawdzający, czy wartość funkcji na początku nie jest już wystarfczająco blisko zera i dzięki temu otrzymujemy pierwiastek bez żadnej iteracji pętli. W metodzie siecznych jako pierwiastek otrzymujemy NaN, ponieważ w obliczaniu s, dzielimy przez zero, co później skutkuje mnożeniem Inf i 0.

\section*{Zadanie 4}
\subsection*{Opis problemu}
Wyznaczyć pierwiastek równania $f(x)=\sin x - (\frac{1}{2}x)^{2}$.\\
\subsection*{Wyniki}
\begin{table}[H]
    \centering
    \begin{tabular}{|c|c|c|c|c|}
        \hline
        Metoda & r&f(r)&it&err\\
        \hline
        Bisekcji & 1.9337539672851562& -2.7027680138402843e-7& 16&0\\
        \hline
        Stycznych & 1.933753779789742& -2.2423316314856834e-8& 4& 0\\
        \hline
        Siecznych & 1.933753644474301& 1.564525129449379e-7& 4& 0\\
        \hline
    \end{tabular}
    \caption{Wyniki dla zadania 4}
    \label{tab:Z4}
\end{table}
\subsection*{Wnioski}
Jak widać wszystkie metody poradziły sobie ze znalezieniem przybliżenia pierwiastka funkcji f. Patrząc na wartości funkcji, to najbliżej zera była metoda stycznych.

\section*{Zadanie 5}
\subsection*{Opis problemu}
W tym zadaniu trzeba metodą bisekcji znależć x, dla którego funkcje $f_{1}(x)=3x$ oraz $f_{2}(x)=e^{x}$ się przecinają. Wykonałem to znajdując miejsce zerowe funkcji pomocniczej $f(x) = f_{1}(x)-f_{2}(x)$.
\subsection*{Wyniki}
\begin{table}[H]
    \centering
    \begin{tabular}{|c|c|c|c|c|c|}
        \hline
        Metoda & $f_{1}(x)$&r&f(r)&it&err\\
        \hline
        Bisekcji &1.857421875& 0.619140625& 9.066320343276146e-5& 9&0\\
        \hline
    \end{tabular}
    \caption{Wyniki dla zadania 5}
    \label{tab:Z5}
\end{table}
\subsection*{Wnioski}
Metoda bisekcji znajduje miejsce zerowe $f(x)$, czyli punkt przecięcia $f_{1}(x)$ oraz $f_{2}(x)$.

\section*{Zadanie 6}
\subsection*{Opis problemu}
W zadaniu należy znaleźć miejsca zerowe funkcji $f_{1}(x) = e^{1-x}-1$ oraz $f_{2}(x) = xe^{-x}$ za pomocą poprzednio używanych metod. Należy odpowiednio dobrać przedziały i przybliżenia. Dodatkowo sprawdzić co się stanie gdy w metodzie stycznych dla $f_{1}$ wybierzemy $x_{0}\in (1, \infty ]$, a dla $f_{2}$ $x_{0}>1$ oraz czy można wybrać $x_{0}=1$ dla $f_{2}$.
\subsection*{Wyniki}
\begin{table}[H]
    \begin{tabular}{|c|c|c|c|c|c|}
        \hline
        Funkcja& Metoda&r&f(r)&it&err\\
        \hline
        \multirow{3}{*}{$f_{1}$}&Bisekcja&1.0&0.0&1&0\\
        \cline{2-6}
        &Stycznych&0.9999999998878352&1.1216494399945987e-10&4&0\\
        \cline{2-6}
        &Siecznych&0.9999994102824874&5.897176864610998e-7&4&0\\
        \hline
        \multirow{3}{*}{$f_{2}$}&Bisekcja&0.0&0.0&1&0\\
        \cline{2-6}
        &Stycznych&-3.0642493416461764e-7&-3.0642502806087233e-7&4&0\\
        \cline{2-6}
        &Siecznych&-1.2229958402039555e-7&-1.2229959897758473e-7&6&0\\
        \hline
        \multirow{3}{*}{Testy}&$f_{1}$ z $x_{0}\in (1, \infty ]$&0.9999999984736215&1.5263785790864404e-9&4&0\\
        \cline{2-6}
        &$f_{2}$ z $x_{0}>1$&14.787436802837927&5.594878975694858e-6&10&0\\
        \cline{2-6}
        &$f_{2}$ z $x_{0}=1$&1.0&0.36787944117144233&1&2\\
        \hline
    \end{tabular}
    \caption{Wyniki dla zadania 6}    
    \label{tab:Z6}
        
\end{table}
\subsection*{Wnioski}
Jak widać na \ref{tab:Z6}, Wszystkie metody znajdują przybliżenie pierwiastków funkcji. W testach natomiast ukazują się ciekawe przypadki.\\
W pierwszym teście w tabeli widnieje wynik dla $x_{0}=1.5$. $x_{0}\in (1,4.5)$ daje nam jeszcze przybliżenie miejsca zerowego. $x_{0}\in [4.5,7.6)$ zwraca nam jeszcze wyniki, chociaż wyznacza niesamowicie niskie r oraz osiąga maksymalną liczbę iteracji. $x_{0}\in [7.6,12.6)$ zwraca nam NaN, ponieważ wartość funkcji zwraca Inf. Dla $x_{0}\ge 12.6$ pochodna jest zbyt  bliska zeru, więc metoda od razu zwraca wynik z sygnałem błędu 2.\\
W drugim teście możemy zauważyć, że metoda dla $x_{0}=1.5$ nie zwraca nam bliższego miejsca zerowego $r=0$, tylko szuka miejsca zerowego dla większych x, gdzie funkcja zbiega do 0.\\
W trzecim teście natychmiast dostajemy wynik z sygnalizacją błędu 2, ponieważ dla $x_{0}=1.0$ pochodna funkcji jest równa 0. Taki przypadek udowadnia, że metoda nie jest zbieżna globalnie.
\end{document}